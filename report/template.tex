\documentclass{article}

\usepackage{arxiv}

\usepackage[utf8]{inputenc} % allow utf-8 input
\usepackage[T1]{fontenc}    % use 8-bit T1 fonts
\usepackage{hyperref}       % hyperlinks
\usepackage{url}            % simple URL typesetting
\usepackage{booktabs}       % professional-quality tables
\usepackage{amsfonts}       % blackboard math symbols
\usepackage{nicefrac}       % compact symbols for 1/2, etc.
\usepackage{microtype}      % microtypography
\usepackage{graphicx}
\usepackage[sorting=none]{biblatex}
\usepackage{doi}

\usepackage{amsmath}
\usepackage{amssymb}
\usepackage{amsthm}
\usepackage{bm}
\usepackage[shortlabels]{enumitem}
\usepackage[scr=rsfs]{mathalpha}
\usepackage{mathtools}
\usepackage{xcolor}

\usepackage{cleveref}       % smart cross-referencing

\newtheorem{definition}{Definition}
\newtheorem{proposition}{Proposition}
\newtheorem{theorem}{Theorem}
\newtheorem{lemma}{Lemma}
\newtheorem{corollary}{Corollary}

\DeclareMathOperator{\lwe}{LWE}
\DeclareMathOperator{\rlwe}{RLWE}
\DeclareMathOperator{\enc}{Enc}
\DeclareMathOperator{\dec}{Dec}

\addbibresource{references.bib}

\title{Efficient Deep Learning on Encrypted Data}

% Here you can change the date presented in the paper title
%\date{September 9, 1985}
% Or remove it
%\date{}

\author{
  Ng Wei~En \\
  School of Computing \\
  National University of Singapore \\
  Singapore, 117417 \\
  \texttt{\href{mailto:ngwe@u.nus.edu}{ngwe@u.nus.edu}}
}

% Uncomment to override  the `A preprint' in the header
\renewcommand{\headeright}{Technical Report (Draft)}
\renewcommand{\undertitle}{Technical Report (Draft)}
\renewcommand{\shorttitle}{Efficient Deep Learning on Encrypted Data}

% TODO
%%% Add PDF metadata to help others organize their library
%%% Once the PDF is generated, you can check the metadata with
%%% $ pdfinfo template.pdf
%\hypersetup{
%  pdftitle={A template for the arxiv style},
%  pdfsubject={q-bio.NC, q-bio.QM},
%  pdfauthor={David S.~Hippocampus, Elias D.~Striatum},
%  pdfkeywords={First keyword, Second keyword, More},
%}

\begin{document}
\maketitle

%%\begin{abstract}
%%\end{abstract}

% keywords can be removed
%\keywords{First keyword \and Second keyword \and More}

\section{Fully Homomorphic Encryption}

For simplicity, we restrict our discussion of Fully Homomorphic Encryption (FHE) to over
the discretized torus as used in TFHE \cite{chillotti2020tfhe} and Concrete
\cite{chillotti2020concrete}. Knowledge of undergraduate abstract algebra is assumed.

The definitions in this section are listed in \cite{chillotti2020tfhe},
\cite{chillotti2020concrete} and \cite{joye2021guide}, unless otherwise stated.

\begin{definition}[Torus]
  The real torus is defined as $\mathbb{T} \coloneqq \mathbb{R}/\mathbb{Z}$, i.e. the
  set of real numbers modulo $1$.
\end{definition}

\begin{definition}[$R$-Module \cite{knapp2006basic}]
  If $R$ is a ring, a $R$-module $M$ is an abelian group endowed with an external
  operation $\cdot: R \times M \to M$ satisfying the following properties:
  \begin{enumerate}
    \item $r \cdot (r' \cdot m) = (r \times r') \cdot m$ for all $r, r' \in R,\ m \in
      M$.
    \item $(r + r') \cdot m = r \cdot m + r' \cdot m$ and $r \cdot (m + m') = r \cdot m
      + r \cdot m'$ for all $r, r' \in R,\ m, m' \in M$.
  \end{enumerate}
\end{definition}

Since any abelian group is a $\mathbb{Z}$-module by construction, we may regard
$\mathbb{T}$ as a $\mathbb{Z}$-module. It follows that for all $N, k \in \mathbb{N}$,
$\mathbb{T}_N[X]^k$ is also a $\mathbb{Z}_N[X]$-module.

\subsection{Learning With Errors (LWE) \& Ring-LWE}

Let $\mathbb{B} = \{0, 1\}$ and $q$ be a power of two, known as the \emph{plaintext
modulus}. %TODO: check this

\begin{definition}[LWE Plaintext]
  The LWE plaintext space is defined as $\mathbb{Z}_q = \mathbb{Z}/q\mathbb{Z}$.
\end{definition}

Let $n \in \mathbb{N}$, known as the \emph{LWE dimension}.

\begin{definition}[LWE Ciphertext]
  For a LWE plaintext $\mu \in \mathbb{Z}_q$ and secret key $\bm{s} = (s_1, \ldots, s_n)
  \in \mathbb{B}^n$, the LWE ciphertext $\bm{c} \leftarrow \lwe_{\bm{s}}(\mu)$ is
  \begin{equation}
    \bm{c} = (a_1, \ldots, a_n, b) \in \mathbb{Z}_q^{n+1},
  \end{equation} where $\bm{a} = (a_1, \ldots, a_n)$ is sampled uniformly at random in
  $\mathbb{Z}_q^n$ and \begin{equation}
    b \coloneqq \langle \bm{a}, \bm{s} \rangle + \mu + e \pmod{q}
  \end{equation} with $e$ a small discretised Gaussian noise.

  %TODO: state clearly the sampling distribution of e
\end{definition}

Given a LWE ciphertext $\bm{c} \leftarrow \lwe_{\bm{s}}(\mu)$ with $\bm{c} = (a_1,
\ldots, a_n, b)$ and the secret key $\bm{s}$, the decryption algorithm computes
\begin{equation}
  \mu^* \coloneqq b - \langle \bm{a}, \bm{s} \rangle = \mu + e \pmod{q}.
\end{equation}

By setting the least significant bits of $\mu$ to zero, for sufficiently small $e$, we
may round off $\mu^*$ to recover the original plaintext $\mu$.

\begin{definition}[Cyclotomic Polynomial]
  For any $n \in \mathbb{N}$, the $n$th cyclotomic polynomial is the unique irreducible
  polynomial that divides $x^n - 1$ and is not a divisor of $x^k - 1$ for any $k < n$.
\end{definition}

Let $N$ be a power of two known as the \emph{RLWE polynomial size}. Note that $x^N + 1$
is the $2N$th cyclotomic polynomial, so we may define $\mathbb{B}_N[X] = \mathbb{B}[X]/
(X^N + 1)$. %TODO: elaborate on what this means

\begin{definition}[RLWE Plaintext]
  The RLWE plaintext space is defined as $\mathbb{Z}_{q,N}[X] = \mathbb{Z}_q[X]/(X^N +
  1)$.
\end{definition}

Let $k \in \mathbb{N}$ be known as the \emph{RLWE dimension}.

\begin{definition}[RLWE Ciphertext]
  For a RLWE plaintext $\mu(X) \in \mathbb{Z}_{q,N}[X]$ and a secret key $\bm{s} =
  (s_1(X), \ldots, s_k(X)) \in \mathbb{B}_N[X]^k$, the RLWE ciphertext $\bm{c}
  \leftarrow \rlwe_{\bm{s}}(\mu(X))$ is \begin{equation}
    \bm{c} = (a_1(X), \ldots, a_k(X), b(X)) \in \mathbb{Z}_{q,N}[X]^{k+1},
  \end{equation} where $\bm{a} = (a_1(X), \ldots, a_k(X))$ is sampled uniformly at
  random in $\mathbb{Z}_{q,N}[X]$ and \begin{equation}
    b(X) \coloneqq \langle \mathbf{a}, \mathbf{s} \rangle + \mu(X) + e(X)
  \end{equation} with $e(X)$ a polynomial with small discretised Gaussian coefficients.
\end{definition}

Given a RLWE ciphertext $\bm{c} \leftarrow \rlwe_{\bm{s}}(\mu(X))$ with $\bm{c} =
(a_1(X), \ldots, a_k(X), b(X))$ and the secret key $\bm{s}$, the decryption algorithm
computes \begin{equation}
  \mu^*(X) \coloneqq b(X) - \langle \bm{a}, \bm{s} \rangle = \mu(X) + e(X)
  \pmod{q, X^N + 1}.
\end{equation}
As for LWE ciphertexts, for sufficiently small coefficients of $e(X)$, we may round off
the coefficients of $\mu^*(X)$ to recover the original plaintext $\mu(X)$.

\subsubsection{Real Encoding}

Concrete encodes messages of a real domain $\mathscr{D} = [0, 1] \subseteq \mathbb{R}$
into a subset of $\mathbb{Z}_q$ with a precision-padding model: \begin{enumerate}
  \item $n_{\text{msg}}$ is the no. of \emph{precision} bits in the MSBs of the
    plaintext, reserved for storing the message $m \in \mathscr{D}$.
  \item $n_{\text{pad}}$ is the no. of \emph{padding} bits in MSBs of the plaintext.
  \item $\sigma \coloneqq 1/(2^{n_{\text{msg}}} - 1)$ is the \emph{granularity/margin}
    which prevents wrong decryption near $0 \in \mathbb{Z}_q$. This effectively reduces
    the domain to $[0, 1 - \sigma)$.
  \item $\delta \coloneqq 1 + \sigma$.
\end{enumerate}

We may now define the following functions $\enc: \mathscr{D} \to \mathbb{Z}_q$ and
$\dec: \mathbb{Z}_q \to \mathscr{D}$: \begin{align}
  \enc(m) &= \frac{q}{2^{n_{\text{msg}} + n_{\text{pad}}}}
  \lceil m \cdot 2^{n_{\text{msg}}} / \delta \rfloor, \\
  \dec(\mu^*) &= \mu^* \cdot \delta \cdot 2^{n_{\text{pad}}}/q.
\end{align}

For scaling to $\mathscr{D} = [a, b] \subseteq \mathbb{R}$, it suffices to scale
$\sigma$ and $\delta$ accordingly.

\subsubsection{Basic Operations}

%TODO

%TODO

\printbibliography

\end{document}
